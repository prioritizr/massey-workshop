\documentclass[12pt,]{book}
\usepackage{lmodern}
\usepackage{amssymb,amsmath}
\usepackage{ifxetex,ifluatex}
\usepackage{fixltx2e} % provides \textsubscript
\ifnum 0\ifxetex 1\fi\ifluatex 1\fi=0 % if pdftex
  \usepackage[T1]{fontenc}
  \usepackage[utf8]{inputenc}
\else % if luatex or xelatex
  \ifxetex
    \usepackage{mathspec}
  \else
    \usepackage{fontspec}
  \fi
  \defaultfontfeatures{Ligatures=TeX,Scale=MatchLowercase}
\fi
% use upquote if available, for straight quotes in verbatim environments
\IfFileExists{upquote.sty}{\usepackage{upquote}}{}
% use microtype if available
\IfFileExists{microtype.sty}{%
\usepackage[]{microtype}
\UseMicrotypeSet[protrusion]{basicmath} % disable protrusion for tt fonts
}{}
\PassOptionsToPackage{hyphens}{url} % url is loaded by hyperref
\usepackage[unicode=true]{hyperref}
\PassOptionsToPackage{usenames,dvipsnames}{color} % color is loaded by hyperref
\hypersetup{
            pdftitle={PRIORITIZR WORKSHOP MANUAL},
            pdfauthor={Jeffrey O. Hanson},
            colorlinks=true,
            linkcolor=Maroon,
            citecolor=Blue,
            urlcolor=blue,
            breaklinks=true}
\urlstyle{same}  % don't use monospace font for urls
\usepackage[left=2.54cm, right=2.54cm, top=2.54cm, bottom=2.54cm]{geometry}
\usepackage{natbib}
\bibliographystyle{plainnat}
\usepackage{color}
\usepackage{fancyvrb}
\newcommand{\VerbBar}{|}
\newcommand{\VERB}{\Verb[commandchars=\\\{\}]}
\DefineVerbatimEnvironment{Highlighting}{Verbatim}{commandchars=\\\{\}}
% Add ',fontsize=\small' for more characters per line
\usepackage{framed}
\definecolor{shadecolor}{RGB}{248,248,248}
\newenvironment{Shaded}{\begin{snugshade}}{\end{snugshade}}
\newcommand{\KeywordTok}[1]{\textcolor[rgb]{0.13,0.29,0.53}{\textbf{#1}}}
\newcommand{\DataTypeTok}[1]{\textcolor[rgb]{0.13,0.29,0.53}{#1}}
\newcommand{\DecValTok}[1]{\textcolor[rgb]{0.00,0.00,0.81}{#1}}
\newcommand{\BaseNTok}[1]{\textcolor[rgb]{0.00,0.00,0.81}{#1}}
\newcommand{\FloatTok}[1]{\textcolor[rgb]{0.00,0.00,0.81}{#1}}
\newcommand{\ConstantTok}[1]{\textcolor[rgb]{0.00,0.00,0.00}{#1}}
\newcommand{\CharTok}[1]{\textcolor[rgb]{0.31,0.60,0.02}{#1}}
\newcommand{\SpecialCharTok}[1]{\textcolor[rgb]{0.00,0.00,0.00}{#1}}
\newcommand{\StringTok}[1]{\textcolor[rgb]{0.31,0.60,0.02}{#1}}
\newcommand{\VerbatimStringTok}[1]{\textcolor[rgb]{0.31,0.60,0.02}{#1}}
\newcommand{\SpecialStringTok}[1]{\textcolor[rgb]{0.31,0.60,0.02}{#1}}
\newcommand{\ImportTok}[1]{#1}
\newcommand{\CommentTok}[1]{\textcolor[rgb]{0.56,0.35,0.01}{\textit{#1}}}
\newcommand{\DocumentationTok}[1]{\textcolor[rgb]{0.56,0.35,0.01}{\textbf{\textit{#1}}}}
\newcommand{\AnnotationTok}[1]{\textcolor[rgb]{0.56,0.35,0.01}{\textbf{\textit{#1}}}}
\newcommand{\CommentVarTok}[1]{\textcolor[rgb]{0.56,0.35,0.01}{\textbf{\textit{#1}}}}
\newcommand{\OtherTok}[1]{\textcolor[rgb]{0.56,0.35,0.01}{#1}}
\newcommand{\FunctionTok}[1]{\textcolor[rgb]{0.00,0.00,0.00}{#1}}
\newcommand{\VariableTok}[1]{\textcolor[rgb]{0.00,0.00,0.00}{#1}}
\newcommand{\ControlFlowTok}[1]{\textcolor[rgb]{0.13,0.29,0.53}{\textbf{#1}}}
\newcommand{\OperatorTok}[1]{\textcolor[rgb]{0.81,0.36,0.00}{\textbf{#1}}}
\newcommand{\BuiltInTok}[1]{#1}
\newcommand{\ExtensionTok}[1]{#1}
\newcommand{\PreprocessorTok}[1]{\textcolor[rgb]{0.56,0.35,0.01}{\textit{#1}}}
\newcommand{\AttributeTok}[1]{\textcolor[rgb]{0.77,0.63,0.00}{#1}}
\newcommand{\RegionMarkerTok}[1]{#1}
\newcommand{\InformationTok}[1]{\textcolor[rgb]{0.56,0.35,0.01}{\textbf{\textit{#1}}}}
\newcommand{\WarningTok}[1]{\textcolor[rgb]{0.56,0.35,0.01}{\textbf{\textit{#1}}}}
\newcommand{\AlertTok}[1]{\textcolor[rgb]{0.94,0.16,0.16}{#1}}
\newcommand{\ErrorTok}[1]{\textcolor[rgb]{0.64,0.00,0.00}{\textbf{#1}}}
\newcommand{\NormalTok}[1]{#1}
\usepackage{longtable,booktabs}
% Fix footnotes in tables (requires footnote package)
\IfFileExists{footnote.sty}{\usepackage{footnote}\makesavenoteenv{long table}}{}
\usepackage{graphicx,grffile}
\makeatletter
\def\maxwidth{\ifdim\Gin@nat@width>\linewidth\linewidth\else\Gin@nat@width\fi}
\def\maxheight{\ifdim\Gin@nat@height>\textheight\textheight\else\Gin@nat@height\fi}
\makeatother
% Scale images if necessary, so that they will not overflow the page
% margins by default, and it is still possible to overwrite the defaults
% using explicit options in \includegraphics[width, height, ...]{}
\setkeys{Gin}{width=\maxwidth,height=\maxheight,keepaspectratio}
\IfFileExists{parskip.sty}{%
\usepackage{parskip}
}{% else
\setlength{\parindent}{0pt}
\setlength{\parskip}{6pt plus 2pt minus 1pt}
}
\setlength{\emergencystretch}{3em}  % prevent overfull lines
\providecommand{\tightlist}{%
  \setlength{\itemsep}{0pt}\setlength{\parskip}{0pt}}
\setcounter{secnumdepth}{5}
% Redefines (sub)paragraphs to behave more like sections
\ifx\paragraph\undefined\else
\let\oldparagraph\paragraph
\renewcommand{\paragraph}[1]{\oldparagraph{#1}\mbox{}}
\fi
\ifx\subparagraph\undefined\else
\let\oldsubparagraph\subparagraph
\renewcommand{\subparagraph}[1]{\oldsubparagraph{#1}\mbox{}}
\fi

% set default figure placement to htbp
\makeatletter
\def\fps@figure{htbp}
\makeatother

% load packages
\usepackage{caption}
\usepackage{float}

% default bookdown preamble
\usepackage{booktabs}
\usepackage{amsthm}
\makeatletter
\def\thm@space@setup{%
  \thm@preskip=8pt plus 2pt minus 4pt
  \thm@postskip=\thm@preskip
}
\makeatother

% remove figure labelling
\captionsetup[figure]{labelformat=empty,textfont=it}

% make figures static
\let\origfigure\figure
\let\endorigfigure\endfigure
\renewenvironment{figure}[1][2] {
  \expandafter\origfigure\expandafter[H]
} {
  \endorigfigure
}

% text boxes
\ifxetex
  \usepackage{letltxmacro}
  \setlength{\XeTeXLinkMargin}{1pt}
  \LetLtxMacro\SavedIncludeGraphics\includegraphics
  \def\includegraphics#1#{% #1 catches optional stuff (star/opt. arg.)
    \IncludeGraphicsAux{#1}%
  }%
  \newcommand*{\IncludeGraphicsAux}[2]{%
    \XeTeXLinkBox{%
      \SavedIncludeGraphics#1{#2}%
    }%
  }%
\fi

\makeatletter
\newenvironment{kframe}{%
\medskip{}
\setlength{\fboxsep}{.8em}
 \def\at@end@of@kframe{}%
 \ifinner\ifhmode%
  \def\at@end@of@kframe{\end{minipage}}%
  \begin{minipage}{\columnwidth}%
 \fi\fi%
 \def\FrameCommand##1{\hskip\@totalleftmargin \hskip-\fboxsep
 \colorbox{shadecolor}{##1}\hskip-\fboxsep
     % There is no \\@totalrightmargin, so:
     \hskip-\linewidth \hskip-\@totalleftmargin \hskip\columnwidth}%
 \MakeFramed {\advance\hsize-\width
   \@totalleftmargin\z@ \linewidth\hsize
   \@setminipage}}%
 {\par\unskip\endMakeFramed%
 \at@end@of@kframe}
\makeatother

\makeatletter
\@ifundefined{Shaded}{
}{\renewenvironment{Shaded}{\begin{kframe}}{\end{kframe}}}
\makeatother

\newenvironment{rmdblock}[1]
  {
  \begin{itemize}
  \renewcommand{\labelitemi}{
    \raisebox{-.7\height}[0pt][0pt]{
      {\setkeys{Gin}{width=3em,keepaspectratio}\includegraphics{images/#1}}
    }
  }
  \setlength{\fboxsep}{1em}
  \begin{kframe}
  \item
  }
  {
  \end{kframe}
  \end{itemize}
  }
\newenvironment{rmdnote}
  {\begin{rmdblock}{note}}
  {\end{rmdblock}}
\newenvironment{rmdcaution}
  {\begin{rmdblock}{caution}}
  {\end{rmdblock}}
\newenvironment{rmdquestion}
  {\begin{rmdblock}{question}}
  {\end{rmdblock}}
\newenvironment{rmdanswer}
  {\begin{rmdblock}{answer}}
  {\end{rmdblock}}
\newenvironment{rmdimportant}
  {\begin{rmdblock}{important}}
  {\end{rmdblock}}
\newenvironment{rmdtip}
  {\begin{rmdblock}{tip}}
  {\end{rmdblock}}
\newenvironment{rmdwarning}
  {\begin{rmdblock}{warning}}
  {\end{rmdblock}}

\title{PRIORITIZR WORKSHOP MANUAL}
\author{Jeffrey O. Hanson}
\date{2020-07-21}

\begin{document}
\maketitle

{
\hypersetup{linkcolor=black}
\setcounter{tocdepth}{0}
\tableofcontents
}
\chapter{Welcome!}\label{welcome}

Here you will find the manual for the prioritizr workshop held at Massey
University, Palmerston North, New Zealand. \textbf{Before you arrive at
the workshop, you should make sure that you have correctly
\protect\hyperlink{setup}{set up your computer for the workshop} and you
have
\href{https://github.com/prioritizr/massey-workshop/raw/master/data.zip}{downloaded
the data from here}. Since we cannot guarantee a stable Internet
connection during the workshop, you may be unable to complete the
workshop if you have not set up your computer beforehand.}

\chapter{Introduction}\label{introduction}

\section{Overview}\label{overview}

The aim of this workshop is to help you get started with using the
prioritizr R package for systematic conservation planning. It is not
designed to give you a comprehensive overview and you will not become an
expert after completing this workshop. Instead, we want to help you
understand the core principles of conservation planning and guide you
through some of the common tasks involved with developing
prioritizations. In other words, we want to give you the knowledge base
and confidence needed to start applying systematic conservation planning
to your own work.

You are not alone in this workshop. If you are having trouble, please
put your hand up and one of the instructors will help you as soon as
they can. You can also ask the people sitting next to you for help too.
\textbf{Most importantly, the code needed to answer the questions in
this workshop are almost always located in the same section as the
question. So if you are stuck, try rereading the example code and see if
you can modify it to answer the question.} Please note that the first
thing an instructor will ask you will be ``what have you tried so
far?''. We can't help you if you haven't tried anything.

\hypertarget{setup}{\section{Setting up your computer}\label{setup}}

You will need to have both \href{https://www.r-project.org}{R} and
\href{https://www.rstudio.com/}{RStudio} installed on your computer to
complete this workshop. Although it is not imperative that you have the
latest version of RStudio installed, \textbf{you will need the latest
version of R installed (i.e.~version 4.0.0)}. Please note that you might
need administrative permissions to install these programs. After
installing them, you will also need to install some R packages too.

\subsection{R}\label{r}

The \href{https://www.r-project.org}{R statistical computing
environment} can be downloaded from the Comprehensive R Archive Network
(CRAN). Specifically, you can download the latest version of R (version
4.0.0) from here: \url{https://cloud.r-project.org}. Please note that
you will need to download the correct file for your operating system
(i.e.~Linux, Mac OSX, Windows).

\subsection{RStudio}\label{rstudio}

\href{https://www.rstudio.com}{RStudio} is an integrated development
environment (IDE). In other words, it is a program that is designed to
make your R programming experience more enjoyable. During this workshop,
you will interact with R through RStudio---meaning that you will open
RStudio to code in R. You can download the latest version of RStudio
here: \url{http://www.rstudio.com/download}. When you start RStudio, you
will see two main parts of the interface:

\begin{center}\includegraphics[width=1\linewidth]{images/rstudio-console} \end{center}

You can type R code into the \emph{Console} and press the enter key to
run code.

\subsection{R packages}\label{r-packages}

An R package is a collection of R code and documentation that can be
installed to enhance the standard R environment with additional
functionality. Currently, there are over fifteen thousand R packages
available on CRAN. Each of these R packages are developed to perform a
specific task, such as
\href{https://cran.r-project.org/web/packages/readxl/index.html}{reading
Excel spreadsheets},
\href{https://cran.r-project.org/web/packages/MODIStsp/index.html}{downloading
satellite imagery data},
\href{https://cran.r-project.org/web/packages/wdpar/index.html}{downloading
and cleaning protected area data}, or
\href{https://cran.r-project.org/web/packages/ENMeval/index.html}{fitting
environmental niche models}. In fact, R has such a diverse ecosystem of
R packages, that the question is almost always not ``can I use R to
\ldots{}?'' but ``what R package can I use to \ldots{}?''. During this
workshop, we will use several R packages. To install these R packages,
please enter the code below in the \emph{Console} part of the RStudio
interface and press enter. Note that you will require an Internet
connection and the installation process may take some time to complete.

\begin{Shaded}
\begin{Highlighting}[]
\KeywordTok{install.packages}\NormalTok{(}\KeywordTok{c}\NormalTok{(}\StringTok{"sf"}\NormalTok{, }\StringTok{"dplyr"}\NormalTok{, }\StringTok{"sp"}\NormalTok{, }\StringTok{"rgeos"}\NormalTok{, }\StringTok{"rgdal"}\NormalTok{, }\StringTok{"raster"}\NormalTok{,}
                   \StringTok{"units"}\NormalTok{, }\StringTok{"prioritizr"}\NormalTok{, }\StringTok{"prioritizrdata"}\NormalTok{, }\StringTok{"Rsymphony"}\NormalTok{,}
                   \StringTok{"mapview"}\NormalTok{, }\StringTok{"assertthat"}\NormalTok{, }\StringTok{"remotes"}\NormalTok{, }\StringTok{"gridExtra"}\NormalTok{,}
                   \StringTok{"BiocManager"}\NormalTok{))}
\NormalTok{BiocManager}\OperatorTok{::}\KeywordTok{install}\NormalTok{(}\StringTok{"lpsymphony"}\NormalTok{)}
\end{Highlighting}
\end{Shaded}

\section{Further reading}\label{further-reading}

There is a wealth of resources available for learning how to use R.
Although not required for this workshop, I would highly recommend that
you read \href{https://r4ds.had.co.nz/}{\emph{R for Data Science} by
Garrett Grolemund and Hadley Wickham}. \textbf{This veritable trove of R
goodness is freely available online.} If you spend a week going through
this book then you will save months debugging and rerunning incorrect
code. I would urge any and all ecologists, especially those working on
Masters or PhD degrees, to read this book. I even bought this book as a
Christmas present for my sister---and, yes, she was happy to receive it!
For intermediate users looking to skill-up, I would recommend the
\href{http://shop.oreilly.com/product/9781593273842.do}{\emph{The Art of
R Programming: A Tour of Statistical Software Design} by Norman Matloff}
and \href{https://adv-r.hadley.nz/}{\emph{Advanced R} by Hadley
Wickham}. Finally, if you wish to learn more about using R as a
geospatial information system (GIS), I would recommend
\href{https://geocompr.robinlovelace.net/}{\emph{Geocomputation with R}
by Robin Lovelace, Jakub Nowosad, and Jannes Muenchow} which is also
freely available online. I also recommend
\href{https://www.springer.com/gp/book/9781461476177}{\emph{Applied
Spatial Data Analysis} by Roger S. Bivand, Edzer Pebesma, and Virgilio
Gómez-Rubio} too.

\chapter{Data}\label{data}

\section{Starting out}\label{starting-out}

We will start by opening RStudio. Ideally, you will have already
installed both R and Rstudio before the workshop. If you have not done
this already, then please see the \protect\hyperlink{setup}{Setting up
your computer} section. \textbf{During this workshop, please do not copy
and paste code from the workshop manual into RStudio. Instead, please
write it out yourself in an R script.} When programming, you will spend
a lot of time fixing coding mistakes---that is, debugging your code---so
it is best to get used to making mistakes now when you have people here
to help you. You can create a new R script by clicking on \emph{File} in
the RStudio menu bar, then \emph{New File}, and then \emph{R Script}.

\begin{center}\includegraphics[width=0.7\linewidth]{images/rstudio-new-script} \end{center}

After creating a new script, you will notice that a new \emph{Source}
panel has appeared. In the \emph{Source} panel, you can type and edit
code before you run it. You can run code in the \emph{Source} panel by
placing the cursor (i.e.~the blinking line) on the desired line of code
and pressing \texttt{Control\ +\ Enter} on your keyboard (or
\texttt{CMD\ +\ Enter} if you are using an Apple computer). You can save
the code in the \emph{Source} panel by pressing \texttt{Control\ +\ s}
on your keyboard (or \texttt{CMD\ +\ s} if you are using an Apple
computer).

\begin{center}\includegraphics[width=0.7\linewidth]{images/rstudio-source} \end{center}

You can also make notes and write your answers to the workshop questions
inside the R script. When writing notes and answers, add a \texttt{\#}
symbol so that the text following the \texttt{\#} symbol is treated as a
comment and not code. This means that you don't have to worry about
highlighting specific parts of the script to avoid errors.

\begin{Shaded}
\begin{Highlighting}[]
\CommentTok{# this is a comment and R will ignore this text if you run it}
\CommentTok{# R will run the code below because it does not start with a # symbol}
\KeywordTok{print}\NormalTok{(}\StringTok{"this is not a comment"}\NormalTok{)}
\end{Highlighting}
\end{Shaded}

\begin{verbatim}
## [1] "this is not a comment"
\end{verbatim}

\begin{Shaded}
\begin{Highlighting}[]
\CommentTok{# you can also add comments to the same line of R code too}
\KeywordTok{print}\NormalTok{(}\StringTok{"this is also not a comment"}\NormalTok{) }\CommentTok{# but this is a comment}
\end{Highlighting}
\end{Shaded}

\begin{verbatim}
## [1] "this is also not a comment"
\end{verbatim}

\textbf{Remember to save your script regularly to ensure that you don't
lose anything in the event that RStudio crashes (e.g.~using
\texttt{Control\ +\ s} or \texttt{CMD\ +\ s})!}

\section{Attaching packages}\label{attaching-packages}

Now we will set up our R session for the workshop. Specifically, enter
the following R code to attach the R packages used in this workshop.

\begin{Shaded}
\begin{Highlighting}[]
\CommentTok{# load packages}
\KeywordTok{library}\NormalTok{(prioritizr)}
\KeywordTok{library}\NormalTok{(sf)}
\KeywordTok{library}\NormalTok{(rgdal)}
\KeywordTok{library}\NormalTok{(raster)}
\KeywordTok{library}\NormalTok{(rgeos)}
\KeywordTok{require}\NormalTok{(mapview)}
\KeywordTok{library}\NormalTok{(units)}
\KeywordTok{library}\NormalTok{(scales)}
\KeywordTok{library}\NormalTok{(assertthat)}
\KeywordTok{library}\NormalTok{(gridExtra)}
\KeywordTok{library}\NormalTok{(dplyr)}
\end{Highlighting}
\end{Shaded}

You should have already downloaded the data. If you have not already
done so, you can download it from here:
\url{https://github.com/prioritizr/cibio-workshop/raw/master/data.zip}.
After downloading the data, you can unzip the data into a new folder.
Next, you will need to set the working directory to this new folder. To
achieve this, click on the \emph{Session} button on the RStudio menu
bar, then click \emph{Set Working Directory}, and then \emph{Choose
Directory}.

\begin{center}\includegraphics[width=0.7\linewidth]{images/rstudio-wd} \end{center}

\clearpage

Now navigate to the folder where you unzipped the data and select
\emph{Open}. You can verify that you have correctly set the working
directory using the following R code. You should see the output
\texttt{TRUE} in the \emph{Console} panel.

\begin{Shaded}
\begin{Highlighting}[]
\KeywordTok{file.exists}\NormalTok{(}\StringTok{"data/pu.shp"}\NormalTok{)}
\end{Highlighting}
\end{Shaded}

\begin{verbatim}
## [1] TRUE
\end{verbatim}

\section{Data import}\label{data-import}

Now that we have downloaded the dataset, we will need to import it into
our R session. Specifically, this data was obtained from the
``Introduction to Marxan'' course and was originally a subset of a
larger spatial prioritization project performed under contract to
Australia's Department of Environment and Water Resources. It contains
vector-based planning unit data (\texttt{pu.shp}) and the raster-based
data describing the spatial distributions of 33 vegetation classes
(\texttt{vegetation.tif}) in southern Tasmania, Australia. Please note
this dataset is only provided for teaching purposes and should not be
used for any real-world conservation planning. We can import the data
into our R session using the following code.

\begin{Shaded}
\begin{Highlighting}[]
\CommentTok{# import planning unit data}
\NormalTok{pu_data <-}\StringTok{ }\KeywordTok{as}\NormalTok{(}\KeywordTok{read_sf}\NormalTok{(}\StringTok{"data/pu.shp"}\NormalTok{), }\StringTok{"Spatial"}\NormalTok{)}

\CommentTok{# format columns in planning unit data}
\NormalTok{pu_data}\OperatorTok{$}\NormalTok{locked_in <-}\StringTok{ }\KeywordTok{as.logical}\NormalTok{(pu_data}\OperatorTok{$}\NormalTok{locked_in)}
\NormalTok{pu_data}\OperatorTok{$}\NormalTok{locked_out <-}\StringTok{ }\KeywordTok{as.logical}\NormalTok{(pu_data}\OperatorTok{$}\NormalTok{locked_out)}

\CommentTok{# import vegetation data}
\NormalTok{veg_data <-}\StringTok{ }\KeywordTok{stack}\NormalTok{(}\StringTok{"data/vegetation.tif"}\NormalTok{)}
\end{Highlighting}
\end{Shaded}

\clearpage

\section{Planning unit data}\label{planning-unit-data}

The planning unit data contains spatial data describing the geometry for
each planning unit and attribute data with information about each
planning unit (e.g.~cost values). Let's investigate the
\texttt{pu\_data} object. The attribute data contains 5 columns with
contain the following information:

\begin{itemize}
\tightlist
\item
  \texttt{id}: unique identifiers for each planning unit
\item
  \texttt{cost}: acquisition cost values for each planning unit
  (millions of Australian dollars).
\item
  \texttt{status}: status information for each planning unit (only
  relevant with Marxan)
\item
  \texttt{locked\_in}: logical values (i.e.
  \texttt{TRUE}/\texttt{FALSE}) indicating if planning units are covered
  by protected areas or not.
\item
  \texttt{locked\_out}: logical values (i.e.
  \texttt{TRUE}/\texttt{FALSE}) indicating if planning units cannot be
  managed as a protected area because they contain are too degraded.
\end{itemize}

\begin{Shaded}
\begin{Highlighting}[]
\CommentTok{# print a short summary of the data}
\KeywordTok{print}\NormalTok{(pu_data)}
\end{Highlighting}
\end{Shaded}

\begin{verbatim}
## class       : SpatialPolygonsDataFrame 
## features    : 557 
## extent      : 1115191, 1385011, -4840595, -4662354  (xmin, xmax, ymin, ymax)
## crs         : +proj=aea +lat_1=-18 +lat_2=-36 +lat_0=0 +lon_0=132 +x_0=0 +y_0=0 +ellps=GRS80 +units=m +no_defs 
## variables   : 5
## names       :   id,            cost, status, locked_in, locked_out 
## min values  :  574, 2.5534941249227,      0,         0,          0 
## max values  : 1130, 47.238336402701,      2,         1,          1
\end{verbatim}

\begin{Shaded}
\begin{Highlighting}[]
\CommentTok{# plot the planning unit data}
\KeywordTok{plot}\NormalTok{(pu_data)}
\end{Highlighting}
\end{Shaded}

\begin{center}\includegraphics{prioritizr-workshop-manual_files/figure-latex/unnamed-chunk-16-1} \end{center}

\begin{Shaded}
\begin{Highlighting}[]
\CommentTok{# plot an interactive map of the planning unit data}
\KeywordTok{mapview}\NormalTok{(pu_data)}
\end{Highlighting}
\end{Shaded}

\begin{Shaded}
\begin{Highlighting}[]
\CommentTok{# print the structure of object}
\KeywordTok{str}\NormalTok{(pu_data, }\DataTypeTok{max.level =} \DecValTok{2}\NormalTok{)}
\end{Highlighting}
\end{Shaded}

\begin{verbatim}
## Formal class 'SpatialPolygonsDataFrame' [package "sp"] with 5 slots
##   ..@ data       :'data.frame':  557 obs. of  5 variables:
##   ..@ polygons   :List of 557
##   ..@ plotOrder  : int [1:557] 103 219 21 314 206 131 316 327 231 218 ...
##   ..@ bbox       : num [1:2, 1:2] 1115191 -4840595 1385011 -4662354
##   .. ..- attr(*, "dimnames")=List of 2
##   ..@ proj4string:Formal class 'CRS' [package "sp"] with 1 slot
\end{verbatim}

\begin{Shaded}
\begin{Highlighting}[]
\CommentTok{# print the class of the object}
\KeywordTok{class}\NormalTok{(pu_data)}
\end{Highlighting}
\end{Shaded}

\begin{verbatim}
## [1] "SpatialPolygonsDataFrame"
## attr(,"package")
## [1] "sp"
\end{verbatim}

\begin{Shaded}
\begin{Highlighting}[]
\CommentTok{# print the slots of the object}
\KeywordTok{slotNames}\NormalTok{(pu_data)}
\end{Highlighting}
\end{Shaded}

\begin{verbatim}
## [1] "data"        "polygons"    "plotOrder"   "bbox"        "proj4string"
\end{verbatim}

\begin{Shaded}
\begin{Highlighting}[]
\CommentTok{# print the coordinate reference system}
\KeywordTok{print}\NormalTok{(pu_data}\OperatorTok{@}\NormalTok{proj4string)}
\end{Highlighting}
\end{Shaded}

\begin{verbatim}
## CRS arguments:
##  +proj=aea +lat_1=-18 +lat_2=-36 +lat_0=0 +lon_0=132 +x_0=0 +y_0=0
## +ellps=GRS80 +units=m +no_defs
\end{verbatim}

\begin{Shaded}
\begin{Highlighting}[]
\CommentTok{# print number of planning units (geometries) in the data}
\KeywordTok{nrow}\NormalTok{(pu_data)}
\end{Highlighting}
\end{Shaded}

\begin{verbatim}
## [1] 557
\end{verbatim}

\begin{Shaded}
\begin{Highlighting}[]
\CommentTok{# print the first six rows in the data}
\KeywordTok{head}\NormalTok{(pu_data}\OperatorTok{@}\NormalTok{data)}
\end{Highlighting}
\end{Shaded}

\begin{verbatim}
##    id     cost status locked_in locked_out
## 1 574 28.60687      0     FALSE      FALSE
## 2 575 30.83416      0     FALSE      FALSE
## 3 576 38.75511      0     FALSE      FALSE
## 4 577 38.11618      2      TRUE      FALSE
## 5 578 33.33462      2      TRUE      FALSE
## 6 579 42.98948      2      TRUE      FALSE
\end{verbatim}

\begin{Shaded}
\begin{Highlighting}[]
\CommentTok{# print the first six values in the cost column of the attribute data}
\KeywordTok{head}\NormalTok{(pu_data}\OperatorTok{$}\NormalTok{cost)}
\end{Highlighting}
\end{Shaded}

\begin{verbatim}
## [1] 28.60687 30.83416 38.75511 38.11618 33.33462 42.98948
\end{verbatim}

\begin{Shaded}
\begin{Highlighting}[]
\CommentTok{# print the highest cost value}
\KeywordTok{max}\NormalTok{(pu_data}\OperatorTok{$}\NormalTok{cost)}
\end{Highlighting}
\end{Shaded}

\begin{verbatim}
## [1] 47.23834
\end{verbatim}

\begin{Shaded}
\begin{Highlighting}[]
\CommentTok{# print the smallest cost value}
\KeywordTok{min}\NormalTok{(pu_data}\OperatorTok{$}\NormalTok{cost)}
\end{Highlighting}
\end{Shaded}

\begin{verbatim}
## [1] 2.553494
\end{verbatim}

\begin{Shaded}
\begin{Highlighting}[]
\CommentTok{# print average cost value}
\KeywordTok{mean}\NormalTok{(pu_data}\OperatorTok{$}\NormalTok{cost)}
\end{Highlighting}
\end{Shaded}

\begin{verbatim}
## [1] 26.65837
\end{verbatim}

\begin{Shaded}
\begin{Highlighting}[]
\CommentTok{# plot a map of the planning unit cost data}
\KeywordTok{spplot}\NormalTok{(pu_data, }\StringTok{"cost"}\NormalTok{)}
\end{Highlighting}
\end{Shaded}

\begin{center}\includegraphics[width=0.6\linewidth]{prioritizr-workshop-manual_files/figure-latex/unnamed-chunk-18-1} \end{center}

\begin{Shaded}
\begin{Highlighting}[]
\CommentTok{# plot an interactive map of the planning unit cost data}
\KeywordTok{mapview}\NormalTok{(pu_data, }\DataTypeTok{zcol =} \StringTok{"cost"}\NormalTok{)}
\end{Highlighting}
\end{Shaded}

Now, you can try and answer some questions about the planning unit data.

\begin{rmdquestion} 1. How many planning units are in the
planning unit data? 2. What is the highest cost value? 3. Is there a
spatial pattern in the planning unit cost values (hint: use
\texttt{plot} to make a map)?
\end{rmdquestion}

\clearpage

\section{Vegetation data}\label{vegetation-data}

The vegetation data describe the spatial distribution of 33 vegetation
classes in the study area. This data is in a raster format and so the
data are organized using a grid comprising square grid cells that are
each the same size. In our case, the raster data contains multiple
layers (also called ``bands'') and each layer has corresponds to a
spatial grid with exactly the same area and has exactly the same
dimensionality (i.e.~number of rows, columns, and cells). In this
dataset, there are 33 different regular spatial grids layered on top of
each other -- with each layer corresponding to a different vegetation
class -- and each of these layers contains a grid with 171 rows, 319
columns, and 54549 cells. Within each layer, each cell corresponds to a
1 by 1 km square. The values associated with each grid cell indicate the
(one) presence or (zero) absence of a given vegetation class in the
cell.

\begin{figure}
\centering
\includegraphics{images/rasterbands.png}
\caption{}
\end{figure}

Let's explore the vegetation data.

\begin{Shaded}
\begin{Highlighting}[]
\CommentTok{# print a short summary of the data}
\KeywordTok{print}\NormalTok{(veg_data)}
\end{Highlighting}
\end{Shaded}

\begin{verbatim}
## class      : RasterStack 
## dimensions : 171, 319, 54549, 33  (nrow, ncol, ncell, nlayers)
## resolution : 1000, 1000  (x, y)
## extent     : 1080496, 1399496, -4840217, -4669217  (xmin, xmax, ymin, ymax)
## crs        : +proj=aea +lat_1=-18 +lat_2=-36 +lat_0=0 +lon_0=132 +x_0=0 +y_0=0 +ellps=GRS80 +units=m +no_defs 
## names      : vegetation.1, vegetation.2, vegetation.3, vegetation.4, vegetation.5, vegetation.6, vegetation.7, vegetation.8, vegetation.9, vegetation.10, vegetation.11, vegetation.12, vegetation.13, vegetation.14, vegetation.15, ... 
## min values :            0,            0,            0,            0,            0,            0,            0,            0,            0,             0,             0,             0,             0,             0,             0, ... 
## max values :            1,            1,            1,            1,            1,            1,            1,            1,            1,             1,             1,             1,             1,             1,             1, ...
\end{verbatim}

\begin{Shaded}
\begin{Highlighting}[]
\CommentTok{# plot a map of the 20th vegetation class}
\KeywordTok{plot}\NormalTok{(veg_data[[}\DecValTok{20}\NormalTok{]])}
\end{Highlighting}
\end{Shaded}

\begin{center}\includegraphics{prioritizr-workshop-manual_files/figure-latex/explore feature data-1} \end{center}

\begin{Shaded}
\begin{Highlighting}[]
\CommentTok{# plot an interactive map of the 20th vegetation class}
\KeywordTok{mapview}\NormalTok{(veg_data[[}\DecValTok{20}\NormalTok{]])}
\end{Highlighting}
\end{Shaded}

\begin{Shaded}
\begin{Highlighting}[]
\CommentTok{# print number of rows in the data}
\KeywordTok{nrow}\NormalTok{(veg_data)}
\end{Highlighting}
\end{Shaded}

\begin{verbatim}
## [1] 171
\end{verbatim}

\begin{Shaded}
\begin{Highlighting}[]
\CommentTok{# print number of columns  in the data}
\KeywordTok{ncol}\NormalTok{(veg_data)}
\end{Highlighting}
\end{Shaded}

\begin{verbatim}
## [1] 319
\end{verbatim}

\begin{Shaded}
\begin{Highlighting}[]
\CommentTok{# print number of cells in the data}
\KeywordTok{ncell}\NormalTok{(veg_data)}
\end{Highlighting}
\end{Shaded}

\begin{verbatim}
## [1] 54549
\end{verbatim}

\begin{Shaded}
\begin{Highlighting}[]
\CommentTok{# print number of layers in the data}
\KeywordTok{nlayers}\NormalTok{(veg_data)}
\end{Highlighting}
\end{Shaded}

\begin{verbatim}
## [1] 33
\end{verbatim}

\begin{Shaded}
\begin{Highlighting}[]
\CommentTok{# print  resolution on the x-axis}
\KeywordTok{xres}\NormalTok{(veg_data)}
\end{Highlighting}
\end{Shaded}

\begin{verbatim}
## [1] 1000
\end{verbatim}

\begin{Shaded}
\begin{Highlighting}[]
\CommentTok{# print resolution on the y-axis}
\KeywordTok{yres}\NormalTok{(veg_data)}
\end{Highlighting}
\end{Shaded}

\begin{verbatim}
## [1] 1000
\end{verbatim}

\begin{Shaded}
\begin{Highlighting}[]
\CommentTok{# print spatial extent of the grid, i.e. coordinates for corners}
\KeywordTok{extent}\NormalTok{(veg_data)}
\end{Highlighting}
\end{Shaded}

\begin{verbatim}
## class      : Extent 
## xmin       : 1080496 
## xmax       : 1399496 
## ymin       : -4840217 
## ymax       : -4669217
\end{verbatim}

\begin{Shaded}
\begin{Highlighting}[]
\CommentTok{# print the coordinate reference system}
\KeywordTok{print}\NormalTok{(veg_data}\OperatorTok{@}\NormalTok{crs)}
\end{Highlighting}
\end{Shaded}

\begin{verbatim}
## CRS arguments:
##  +proj=aea +lat_1=-18 +lat_2=-36 +lat_0=0 +lon_0=132 +x_0=0 +y_0=0
## +ellps=GRS80 +units=m +no_defs
\end{verbatim}

\begin{Shaded}
\begin{Highlighting}[]
\CommentTok{# print a summary of the first layer in the stack}
\KeywordTok{print}\NormalTok{(veg_data[[}\DecValTok{1}\NormalTok{]])}
\end{Highlighting}
\end{Shaded}

\begin{verbatim}
## class      : RasterLayer 
## band       : 1  (of  33  bands)
## dimensions : 171, 319, 54549  (nrow, ncol, ncell)
## resolution : 1000, 1000  (x, y)
## extent     : 1080496, 1399496, -4840217, -4669217  (xmin, xmax, ymin, ymax)
## crs        : +proj=aea +lat_1=-18 +lat_2=-36 +lat_0=0 +lon_0=132 +x_0=0 +y_0=0 +ellps=GRS80 +units=m +no_defs 
## source     : /home/travis/build/prioritizr/massey-workshop/data/vegetation.tif 
## names      : vegetation.1 
## values     : 0, 1  (min, max)
\end{verbatim}

\begin{Shaded}
\begin{Highlighting}[]
\CommentTok{# print the value in the 800th cell in the first layer of the stack}
\KeywordTok{print}\NormalTok{(veg_data[[}\DecValTok{1}\NormalTok{]][}\DecValTok{800}\NormalTok{])}
\end{Highlighting}
\end{Shaded}

\begin{verbatim}
##   
## 0
\end{verbatim}

\begin{Shaded}
\begin{Highlighting}[]
\CommentTok{# print the value of the cell located in the 30th row and the 60th column of}
\CommentTok{# the first layer}
\KeywordTok{print}\NormalTok{(veg_data[[}\DecValTok{1}\NormalTok{]][}\DecValTok{30}\NormalTok{, }\DecValTok{60}\NormalTok{])}
\end{Highlighting}
\end{Shaded}

\begin{verbatim}
##   
## 0
\end{verbatim}

\begin{Shaded}
\begin{Highlighting}[]
\CommentTok{# calculate the sum of all the cell values in the first layer}
\KeywordTok{cellStats}\NormalTok{(veg_data[[}\DecValTok{1}\NormalTok{]], }\StringTok{"sum"}\NormalTok{)}
\end{Highlighting}
\end{Shaded}

\begin{verbatim}
## [1] 17
\end{verbatim}

\begin{Shaded}
\begin{Highlighting}[]
\CommentTok{# calculate the maximum value of all the cell values in the first layer}
\KeywordTok{cellStats}\NormalTok{(veg_data[[}\DecValTok{1}\NormalTok{]], }\StringTok{"max"}\NormalTok{)}
\end{Highlighting}
\end{Shaded}

\begin{verbatim}
## [1] 1
\end{verbatim}

\begin{Shaded}
\begin{Highlighting}[]
\CommentTok{# calculate the minimum value of all the cell values in the first layer}
\KeywordTok{cellStats}\NormalTok{(veg_data[[}\DecValTok{1}\NormalTok{]], }\StringTok{"min"}\NormalTok{)}
\end{Highlighting}
\end{Shaded}

\begin{verbatim}
## [1] 0
\end{verbatim}

\begin{Shaded}
\begin{Highlighting}[]
\CommentTok{# calculate the mean value of all the cell values in the first layer}
\KeywordTok{cellStats}\NormalTok{(veg_data[[}\DecValTok{1}\NormalTok{]], }\StringTok{"mean"}\NormalTok{)}
\end{Highlighting}
\end{Shaded}

\begin{verbatim}
## [1] 0.0003116464
\end{verbatim}

\clearpage

Now, you can try and answer some questions about the vegetation data.

\begin{rmdquestion} 1. What part of the study area is the 13th
vegetation class found in (hint: make a map)? For instance, is it in the
south-eastern part of the study area? 2. What proportion of cells
contain the 12th vegetation class? 3. Which vegetation class is the most
abundant (i.e.~present in the greatest number of cells)?
\end{rmdquestion}

\chapter{Gap analysis}\label{gap-analysis}

\section{Introduction}\label{introduction}

Before we begin to prioritize areas for protected area establishment, we
should first understand how well existing protected areas are conserving
our biodiversity features (i.e.~native vegetation classes in Tasmania,
Australia). This step is critical: we cannot develop plans to improve
conservation of biodiversity if we don't understand how well existing
policies are currently conserving biodiversity! To achieve this, we can
perform a ``gap analysis''. A gap analysis involves calculating how well
each of our biodiversity features (i.e.~vegetation classes in this
exercise) are represented (covered) by protected areas. Next, we compare
current representation by protected areas of each feature (e.g.~5\% of
their spatial distribution covered by protected areas) to a target
threshold (e.g.~20\% of their spatial distribution covered by protected
areas). This target threshold denotes the minimum amount (e.g.~minimum
proportion of spatial distribution) that we need of each feature to be
represented in the protected area system. Ideally, targets should be
based on an estimate of how much area or habitat is needed for ecosystem
function or species persistence. In practice, targets are generally set
using simple rules of thumb (e.g.~10\% or 20\%), policy (17\%;
\url{https://www.cbd.int/sp/targets/rationale/target-11}) or standard
practices (e.g.~setting targets for species based on geographic range
size) \citep{r1, r2}.

\section{Feature abundance}\label{feature-abundance}

Now we will perform some preliminary calculations to explore the data.
First, we will calculate how much of each vegetation feature occurs
inside each planning unit (i.e.~the abundance of the features). To
achieve this, we will use the \texttt{problem} function to create an
empty conservation planning problem that only contains the planning unit
and biodiversity data. We will then use the \texttt{feature\_abundances}
function to calculate the total amount of each feature in each planning
unit.

\begin{Shaded}
\begin{Highlighting}[]
\CommentTok{# create prioritizr problem with only the data}
\NormalTok{p0 <-}\StringTok{ }\KeywordTok{problem}\NormalTok{(pu_data, veg_data, }\DataTypeTok{cost_column =} \StringTok{"cost"}\NormalTok{)}

\CommentTok{# print empty problem,}
\CommentTok{# we can see that only the cost and feature data are defined}
\KeywordTok{print}\NormalTok{(p0)}
\end{Highlighting}
\end{Shaded}

\begin{verbatim}
## Conservation Problem
##   planning units: SpatialPolygonsDataFrame (557 units)
##   cost:           min: 2.55349, max: 47.23834
##   features:       vegetation.1, vegetation.2, vegetation.3, ... (33 features)
##   objective:      none
##   targets:        none
##   decisions:      default
##   constraints:    <none>
##   penalties:      <none>
##   portfolio:      default
##   solver:         default
\end{verbatim}

\begin{Shaded}
\begin{Highlighting}[]
\CommentTok{# calculate amount of each feature in each planning unit}
\NormalTok{abundance_data <-}\StringTok{ }\KeywordTok{feature_abundances}\NormalTok{(p0)}

\CommentTok{# print abundance data}
\KeywordTok{print}\NormalTok{(abundance_data)}
\end{Highlighting}
\end{Shaded}

\begin{verbatim}
## # A tibble: 33 x 3
##    feature       absolute_abundance relative_abundance
##    <chr>                      <dbl>              <dbl>
##  1 vegetation.1                16.                   1
##  2 vegetation.2                12.2                  1
##  3 vegetation.3                11.4                  1
##  4 vegetation.4                17.0                  1
##  5 vegetation.5                11.1                  1
##  6 vegetation.6                15.2                  1
##  7 vegetation.7                27.1                  1
##  8 vegetation.8                16.1                  1
##  9 vegetation.9                18.0                  1
## 10 vegetation.10               19.5                  1
## # ... with 23 more rows
\end{verbatim}

\clearpage

\begin{Shaded}
\begin{Highlighting}[]
\CommentTok{# note that only the first ten rows are printed,}
\CommentTok{# this is because the abundance_data object is a tibble (i.e. tbl_df) object}
\CommentTok{# and not a standard data.frame object}
\KeywordTok{print}\NormalTok{(}\KeywordTok{class}\NormalTok{(abundance_data))}
\end{Highlighting}
\end{Shaded}

\begin{verbatim}
## [1] "tbl_df"     "tbl"        "data.frame"
\end{verbatim}

\begin{Shaded}
\begin{Highlighting}[]
\CommentTok{# we can print all of the rows in abundance_data like this}
\KeywordTok{print}\NormalTok{(abundance_data, }\DataTypeTok{n =} \OtherTok{Inf}\NormalTok{)}
\end{Highlighting}
\end{Shaded}

\begin{verbatim}
## # A tibble: 33 x 3
##    feature       absolute_abundance relative_abundance
##    <chr>                      <dbl>              <dbl>
##  1 vegetation.1                16.                   1
##  2 vegetation.2                12.2                  1
##  3 vegetation.3                11.4                  1
##  4 vegetation.4                17.0                  1
##  5 vegetation.5                11.1                  1
##  6 vegetation.6                15.2                  1
##  7 vegetation.7                27.1                  1
##  8 vegetation.8                16.1                  1
##  9 vegetation.9                18.0                  1
## 10 vegetation.10               19.5                  1
## 11 vegetation.11               22.5                  1
## 12 vegetation.12              852.                   1
## 13 vegetation.13              178.                   1
## 14 vegetation.14               11.2                  1
## 15 vegetation.15               19.2                  1
## 16 vegetation.16               14.1                  1
## 17 vegetation.17              208.                   1
## 18 vegetation.18               14.4                  1
## 19 vegetation.19               16.0                  1
## 20 vegetation.20               17.9                  1
## 21 vegetation.21               17.5                  1
## 22 vegetation.22              292.                   1
## 23 vegetation.23               21.4                  1
## 24 vegetation.24              164.                   1
## 25 vegetation.25              715.                   1
## 26 vegetation.26               25.0                  1
## 27 vegetation.27               17.7                  1
## 28 vegetation.28               17.7                  1
## 29 vegetation.29               24.2                  1
## 30 vegetation.30               59.6                  1
## 31 vegetation.31               84.0                  1
## 32 vegetation.32               10.0                  1
## 33 vegetation.33               65.0                  1
\end{verbatim}

The \texttt{abundance\_data} object contains three columns. The
\texttt{feature} column contains the name of each feature (derived from
\texttt{names(veg\_data)}), the \texttt{absolute\_abundance} column
contains the total amount of each feature in all the planning units, and
the \texttt{relative\_abundance} column contains the total amount of
each feature in the planning units expressed as a proportion of the
total amount in the underlying raster data. Since all the raster cells
containing vegetation overlap with the planning units, all of the values
in the \texttt{relative\_abundance} column are equal to one (meaning
100\%). Now let's add a new column with the feature abundances expressed
in area units (i.e.~km\textsuperscript{2}).

\begin{Shaded}
\begin{Highlighting}[]
\CommentTok{# add new column with feature abundances in km^2}
\NormalTok{abundance_data}\OperatorTok{$}\NormalTok{absolute_abundance_km2 <-}
\StringTok{  }\NormalTok{(abundance_data}\OperatorTok{$}\NormalTok{absolute_abundance }\OperatorTok{*}\StringTok{ }\KeywordTok{prod}\NormalTok{(}\KeywordTok{res}\NormalTok{(veg_data))) }\OperatorTok
\StringTok{  }\KeywordTok{set_units}\NormalTok{(m}\OperatorTok{^}\DecValTok{2}\NormalTok{) }\OperatorTok
\StringTok{  }\KeywordTok{set_units}\NormalTok{(km}\OperatorTok{^}\DecValTok{2}\NormalTok{)}

\CommentTok{# print abundance data}
\KeywordTok{print}\NormalTok{(abundance_data)}
\end{Highlighting}
\end{Shaded}

\begin{verbatim}
## # A tibble: 33 x 4
##    feature       absolute_abundance relative_abundance absolute_abundance_km2
##    <chr>                      <dbl>              <dbl>                 [km^2]
##  1 vegetation.1                16.                   1               16.00000
##  2 vegetation.2                12.2                  1               12.23304
##  3 vegetation.3                11.4                  1               11.41238
##  4 vegetation.4                17.0                  1               16.97727
##  5 vegetation.5                11.1                  1               11.11602
##  6 vegetation.6                15.2                  1               15.23127
##  7 vegetation.7                27.1                  1               27.11938
##  8 vegetation.8                16.1                  1               16.13828
##  9 vegetation.9                18.0                  1               17.99032
## 10 vegetation.10               19.5                  1               19.54535
## # ... with 23 more rows
\end{verbatim}

Now let's explore the abundance data.

\begin{Shaded}
\begin{Highlighting}[]
\CommentTok{# calculate the average abundance of the features}
\KeywordTok{mean}\NormalTok{(abundance_data}\OperatorTok{$}\NormalTok{absolute_abundance_km2)}
\end{Highlighting}
\end{Shaded}

\begin{verbatim}
## 91.81647 [km^2]
\end{verbatim}

\begin{Shaded}
\begin{Highlighting}[]
\CommentTok{# plot histogram of the features' abundances}
\KeywordTok{hist}\NormalTok{(abundance_data}\OperatorTok{$}\NormalTok{absolute_abundance_km2, }\DataTypeTok{main =} \StringTok{"Feature abundances"}\NormalTok{)}
\end{Highlighting}
\end{Shaded}

\begin{center}\includegraphics{prioritizr-workshop-manual_files/figure-latex/unnamed-chunk-27-1} \end{center}

\begin{Shaded}
\begin{Highlighting}[]
\CommentTok{# find the abundance of the feature with the largest abundance}
\KeywordTok{max}\NormalTok{(abundance_data}\OperatorTok{$}\NormalTok{absolute_abundance_km2)}
\end{Highlighting}
\end{Shaded}

\begin{verbatim}
## 851.9835 [km^2]
\end{verbatim}

\begin{Shaded}
\begin{Highlighting}[]
\CommentTok{# find the name of the feature with the largest abundance}
\NormalTok{abundance_data}\OperatorTok{$}\NormalTok{feature[}\KeywordTok{which.max}\NormalTok{(abundance_data}\OperatorTok{$}\NormalTok{absolute_abundance_km2)]}
\end{Highlighting}
\end{Shaded}

\begin{verbatim}
## [1] "vegetation.12"
\end{verbatim}

Now, try to answer the following questions.

\begin{rmdquestion} 1. What is the median abundance of the
features (hint: \texttt{median})? 2. What is the name of the feature
with smallest abundance? 3. How many features have a total abundance
greater than 100 km\^{}2 (hint: use
\texttt{sum(abundance\_data\$absolute\_abundance\_km2\ \textgreater{}\ set\_units(threshold,\ km\^{}2)}
with the correct \texttt{threshold} value)?
\end{rmdquestion}

\section{Feature representation}\label{feature-representation}

After calculating the total amount of each feature in the planning units
(i.e.~the features' abundances), we will now calculate the amount of
each feature in the planning units that are covered by protected areas
(i.e.~feature representation by protected areas). We can complete this
task using the \texttt{feature\_representation} function. This function
requires (i) a conservation problem object with the planning unit and
biodiversity data and also (ii) an object representing a solution to the
problem (i.e an object in the same format as the planning unit data with
values indicating if the planning units are selected or not).

\begin{Shaded}
\begin{Highlighting}[]
\CommentTok{# create column in planning unit data with binary values (zeros and ones)}
\CommentTok{# indicating if a planning unit is covered by protected areas or not}
\NormalTok{pu_data}\OperatorTok{$}\NormalTok{pa_status <-}\StringTok{ }\KeywordTok{as.numeric}\NormalTok{(pu_data}\OperatorTok{$}\NormalTok{locked_in)}

\CommentTok{# calculate feature representation by protected areas}
\NormalTok{repr_data <-}\StringTok{ }\KeywordTok{feature_representation}\NormalTok{(p0, pu_data[, }\StringTok{"pa_status"}\NormalTok{])}

\CommentTok{# print feature representation data}
\KeywordTok{print}\NormalTok{(repr_data)}
\end{Highlighting}
\end{Shaded}

\begin{verbatim}
## # A tibble: 33 x 3
##    feature       absolute_held relative_held
##    <chr>                 <dbl>         <dbl>
##  1 vegetation.1          0            0     
##  2 vegetation.2          0            0     
##  3 vegetation.3          0            0     
##  4 vegetation.4          0            0     
##  5 vegetation.5          0            0     
##  6 vegetation.6          0            0     
##  7 vegetation.7          0            0     
##  8 vegetation.8          0            0     
##  9 vegetation.9          0.600        0.0334
## 10 vegetation.10         0            0     
## # ... with 23 more rows
\end{verbatim}

Similar to the abundance data before, the \texttt{repr\_data} object
contains three columns. The \texttt{feature} column contains the name of
each feature, the \texttt{absolute\_held} column shows the total amount
of each feature held in the solution (i.e.~the planning units covered by
protected areas), and the \texttt{relative\_held} column shows the
proportion of each feature held in the solution (i.e.~the proportion of
each feature's spatial distribution held in protected areas). Since the
\texttt{absolute\_held} values correspond to the number of grid cells in
the \texttt{veg\_data} object with overlap with protected areas, let's
convert them to area units (i.e.~km\textsuperscript{2}) so we can report
them.

\begin{Shaded}
\begin{Highlighting}[]
\CommentTok{# add new column with the areas represented in km^2}
\NormalTok{repr_data}\OperatorTok{$}\NormalTok{absolute_held_km2 <-}
\StringTok{  }\NormalTok{(repr_data}\OperatorTok{$}\NormalTok{absolute_held }\OperatorTok{*}\StringTok{ }\KeywordTok{prod}\NormalTok{(}\KeywordTok{res}\NormalTok{(veg_data))) }\OperatorTok
\StringTok{  }\KeywordTok{set_units}\NormalTok{(m}\OperatorTok{^}\DecValTok{2}\NormalTok{) }\OperatorTok
\StringTok{  }\KeywordTok{set_units}\NormalTok{(km}\OperatorTok{^}\DecValTok{2}\NormalTok{)}

\CommentTok{# print representation data}
\KeywordTok{print}\NormalTok{(repr_data)}
\end{Highlighting}
\end{Shaded}

\begin{verbatim}
## # A tibble: 33 x 4
##    feature       absolute_held relative_held absolute_held_km2
##    <chr>                 <dbl>         <dbl>            [km^2]
##  1 vegetation.1          0            0              0.0000000
##  2 vegetation.2          0            0              0.0000000
##  3 vegetation.3          0            0              0.0000000
##  4 vegetation.4          0            0              0.0000000
##  5 vegetation.5          0            0              0.0000000
##  6 vegetation.6          0            0              0.0000000
##  7 vegetation.7          0            0              0.0000000
##  8 vegetation.8          0            0              0.0000000
##  9 vegetation.9          0.600        0.0334         0.6001696
## 10 vegetation.10         0            0              0.0000000
## # ... with 23 more rows
\end{verbatim}

Now let's investigate how well the species are represented.

\begin{rmdquestion} 1. What is the average proportion of the
features held in protected areas (hint: use
\texttt{mean(table\$relative\_held)} with the correct \texttt{table}
name)? 2. If we set a target of 10\% coverage by protected areas, how
many features fail to meet this target (hint: use
\texttt{sum(table\$relative\_held\ \textgreater{}=\ target\_value)} with
the correct \texttt{table} name)? 3. If we set a target of 20\% coverage
by protected areas, how many features fail to meet this target? 4. Is
there a relationship between the total abundance of a feature and how
well it is represented by protected areas (hint:
\texttt{plot(abundance\_data\$absolute\_abundance\ \textasciitilde{}\ repr\_data\$relative\_held)})?
\end{rmdquestion}

\chapter{Spatial prioritizations}\label{spatial-prioritizations}

\section{Introduction}\label{introduction}

Here we will develop prioritizations to identify priority areas for
protected area establishment. Its worth noting that prioritizr is a
decision support tool (similar to \href{http://marxan.org/}{Marxan} and
\href{https://www.helsinki.fi/en/researchgroups/digital-geography-lab/software-developed-in-cbig\#section-52992}{Zonation}).
This means that it is designed to help you make decisions---it can't
make decisions for you.

\section{Starting out simple}\label{starting-out-simple}

To start things off, let's keep things simple. Let's create a
prioritization using the
\href{https://prioritizr.net/reference/add_min_set_objective.html}{minimum
set formulation of the reserve selection problem}. This formulation
means that we want a solution that will meet the targets for our
biodiversity features for minimum cost. Here, we will set 5\% targets
for each vegetation class and use the data in the \texttt{cost} column
to specify acquisition costs. Although we strongly recommend using
\href{https://www.gurobi.com/}{Gurobi} to solve problems (with
\href{https://prioritizr.net/reference/add_gurobi_solver.html}{\texttt{add\_gurobi\_solver}}),
we will use the
\href{https://prioritizr.net/reference/add_lpsymphony_solver.html}{lpsymphony
solver} in this workshop since it is easier to install. The Gurobi
solver is much faster than the lpsymphony solver
(\href{https://prioritizr.net/articles/gurobi_installation.html}{see
here for installation instructions}).

\begin{Shaded}
\begin{Highlighting}[]
\CommentTok{# print planning unit data}
\KeywordTok{print}\NormalTok{(pu_data)}
\end{Highlighting}
\end{Shaded}

\begin{verbatim}
## class       : SpatialPolygonsDataFrame 
## features    : 557 
## extent      : 1115191, 1385011, -4840595, -4662354  (xmin, xmax, ymin, ymax)
## crs         : +proj=aea +lat_1=-18 +lat_2=-36 +lat_0=0 +lon_0=132 +x_0=0 +y_0=0 +ellps=GRS80 +units=m +no_defs 
## variables   : 6
## names       :   id,            cost, status, locked_in, locked_out, pa_status 
## min values  :  574, 2.5534941249227,      0,         0,          0,         0 
## max values  : 1130, 47.238336402701,      2,         1,          1,         1
\end{verbatim}

\begin{Shaded}
\begin{Highlighting}[]
\CommentTok{# make prioritization problem}
\NormalTok{p1 <-}\StringTok{ }\KeywordTok{problem}\NormalTok{(pu_data, veg_data, }\DataTypeTok{cost_column =} \StringTok{"cost"}\NormalTok{) }\OperatorTok
\StringTok{      }\KeywordTok{add_min_set_objective}\NormalTok{() }\OperatorTok
\StringTok{      }\KeywordTok{add_relative_targets}\NormalTok{(}\FloatTok{0.05}\NormalTok{) }\OperatorTok\StringTok{ }\CommentTok{# 5% representation targets}
\StringTok{      }\KeywordTok{add_binary_decisions}\NormalTok{() }\OperatorTok
\StringTok{      }\KeywordTok{add_lpsymphony_solver}\NormalTok{()}

\CommentTok{# print problem}
\KeywordTok{print}\NormalTok{(p1)}
\end{Highlighting}
\end{Shaded}

\begin{verbatim}
## Conservation Problem
##   planning units: SpatialPolygonsDataFrame (557 units)
##   cost:           min: 2.55349, max: 47.23834
##   features:       vegetation.1, vegetation.2, vegetation.3, ... (33 features)
##   objective:      Minimum set objective 
##   targets:        Relative targets [targets (min: 0.05, max: 0.05)]
##   decisions:      Binary decision 
##   constraints:    <none>
##   penalties:      <none>
##   portfolio:      default
##   solver:         Lpsymphony [first_feasible (0), gap (0.1), time_limit (-1), verbose (1)]
\end{verbatim}

\begin{Shaded}
\begin{Highlighting}[]
\CommentTok{# solve problem}
\NormalTok{s1 <-}\StringTok{ }\KeywordTok{solve}\NormalTok{(p1)}

\CommentTok{# print solution, the solution_1 column contains the solution values}
\CommentTok{# indicating if a planning unit is (1) selected or (0) not}
\KeywordTok{print}\NormalTok{(s1)}
\end{Highlighting}
\end{Shaded}

\begin{verbatim}
## class       : SpatialPolygonsDataFrame 
## features    : 557 
## extent      : 1115191, 1385011, -4840595, -4662354  (xmin, xmax, ymin, ymax)
## crs         : +proj=aea +lat_1=-18 +lat_2=-36 +lat_0=0 +lon_0=132 +x_0=0 +y_0=0 +ellps=GRS80 +units=m +no_defs 
## variables   : 7
## names       :   id,            cost, status, locked_in, locked_out, pa_status, solution_1 
## min values  :  574, 2.5534941249227,      0,         0,          0,         0,          0 
## max values  : 1130, 47.238336402701,      2,         1,          1,         1,          1
\end{verbatim}

\begin{Shaded}
\begin{Highlighting}[]
\CommentTok{# calculate number of planning units selected in the prioritization}
\KeywordTok{sum}\NormalTok{(s1}\OperatorTok{$}\NormalTok{solution_}\DecValTok{1}\NormalTok{)}
\end{Highlighting}
\end{Shaded}

\begin{verbatim}
## [1] 16
\end{verbatim}

\begin{Shaded}
\begin{Highlighting}[]
\CommentTok{# calculate total cost of the prioritization}
\KeywordTok{sum}\NormalTok{(s1}\OperatorTok{$}\NormalTok{solution_}\DecValTok{1} \OperatorTok{*}\StringTok{ }\NormalTok{s1}\OperatorTok{$}\NormalTok{cost)}
\end{Highlighting}
\end{Shaded}

\begin{verbatim}
## [1] 408.3382
\end{verbatim}

\begin{Shaded}
\begin{Highlighting}[]
\CommentTok{# plot solution}
\CommentTok{# selected = green, not selected = grey}
\KeywordTok{spplot}\NormalTok{(s1, }\StringTok{"solution_1"}\NormalTok{, }\DataTypeTok{col.regions =} \KeywordTok{c}\NormalTok{(}\StringTok{"grey80"}\NormalTok{, }\StringTok{"darkgreen"}\NormalTok{), }\DataTypeTok{main =} \StringTok{"s1"}\NormalTok{,}
       \DataTypeTok{colorkey =} \OtherTok{FALSE}\NormalTok{)}
\end{Highlighting}
\end{Shaded}

\begin{center}\includegraphics[width=0.65\linewidth]{prioritizr-workshop-manual_files/figure-latex/unnamed-chunk-32-1} \end{center}

Now let's examine the solution.

\begin{rmdquestion} 1. How many planing units were selected in
the prioritization? What proportion of planning units were selected in
the prioritization? 2. Is there a pattern in the spatial distribution of
the priority areas? 3. Can you verify that all of the targets were met
in the prioritization (hint:
\texttt{feature\_representation(p1,\ s1{[},\ "solution\_1"{]})})?
\end{rmdquestion}

\section{Adding complexity}\label{adding-complexity}

Our first prioritization suffers many limitations, so let's add
additional constraints to the problem to make it more useful. First,
let's lock in planing units that are already by covered protected areas.
If some vegetation communities are already secured inside existing
protected areas, then we might not need to add as many new protected
areas to the existing protected area system to meet their targets. Since
our planning unit data (\texttt{pu\_da}) already contains this
information in the \texttt{locked\_in} column, we can use this column
name to specify which planning units should be locked in.

\begin{Shaded}
\begin{Highlighting}[]
\CommentTok{# plot locked_in data}
\CommentTok{# TRUE = blue, FALSE = grey}
\KeywordTok{spplot}\NormalTok{(pu_data, }\StringTok{"locked_in"}\NormalTok{, }\DataTypeTok{col.regions =} \KeywordTok{c}\NormalTok{(}\StringTok{"grey80"}\NormalTok{, }\StringTok{"darkblue"}\NormalTok{),}
       \DataTypeTok{main =} \StringTok{"locked_in"}\NormalTok{, }\DataTypeTok{colorkey =} \OtherTok{FALSE}\NormalTok{)}
\end{Highlighting}
\end{Shaded}

\begin{center}\includegraphics[width=0.65\linewidth]{prioritizr-workshop-manual_files/figure-latex/unnamed-chunk-34-1} \end{center}

\begin{Shaded}
\begin{Highlighting}[]
\CommentTok{# make prioritization problem}
\NormalTok{p2 <-}\StringTok{ }\KeywordTok{problem}\NormalTok{(pu_data, veg_data, }\DataTypeTok{cost_column =} \StringTok{"cost"}\NormalTok{) }\OperatorTok
\StringTok{      }\KeywordTok{add_min_set_objective}\NormalTok{() }\OperatorTok
\StringTok{      }\KeywordTok{add_relative_targets}\NormalTok{(}\FloatTok{0.05}\NormalTok{) }\OperatorTok
\StringTok{      }\KeywordTok{add_locked_in_constraints}\NormalTok{(}\StringTok{"locked_in"}\NormalTok{) }\OperatorTok
\StringTok{      }\KeywordTok{add_binary_decisions}\NormalTok{() }\OperatorTok
\StringTok{      }\KeywordTok{add_lpsymphony_solver}\NormalTok{()}

\CommentTok{# print problem}
\KeywordTok{print}\NormalTok{(p2)}
\end{Highlighting}
\end{Shaded}

\begin{verbatim}
## Conservation Problem
##   planning units: SpatialPolygonsDataFrame (557 units)
##   cost:           min: 2.55349, max: 47.23834
##   features:       vegetation.1, vegetation.2, vegetation.3, ... (33 features)
##   objective:      Minimum set objective 
##   targets:        Relative targets [targets (min: 0.05, max: 0.05)]
##   decisions:      Binary decision 
##   constraints:    <Locked in planning units [203 locked units]>
##   penalties:      <none>
##   portfolio:      default
##   solver:         Lpsymphony [first_feasible (0), gap (0.1), time_limit (-1), verbose (1)]
\end{verbatim}

\begin{Shaded}
\begin{Highlighting}[]
\CommentTok{# solve problem}
\NormalTok{s2 <-}\StringTok{ }\KeywordTok{solve}\NormalTok{(p2)}

\CommentTok{# plot solution}
\CommentTok{# selected = green, not selected = grey}
\KeywordTok{spplot}\NormalTok{(s2, }\StringTok{"solution_1"}\NormalTok{, }\DataTypeTok{col.regions =} \KeywordTok{c}\NormalTok{(}\StringTok{"grey80"}\NormalTok{, }\StringTok{"darkgreen"}\NormalTok{), }\DataTypeTok{main =} \StringTok{"s2"}\NormalTok{,}
       \DataTypeTok{colorkey =} \OtherTok{FALSE}\NormalTok{)}
\end{Highlighting}
\end{Shaded}

\begin{center}\includegraphics[width=0.65\linewidth]{prioritizr-workshop-manual_files/figure-latex/unnamed-chunk-35-1} \end{center}

Let's pretend that we talked to an expert on the vegetation communities
in our study system and they recommended that a 10\% target was needed
for each vegetation class. So, equipped with this information, let's set
the targets to 10\%.

\begin{Shaded}
\begin{Highlighting}[]
\CommentTok{# make prioritization problem}
\NormalTok{p3 <-}\StringTok{ }\KeywordTok{problem}\NormalTok{(pu_data, veg_data, }\DataTypeTok{cost_column =} \StringTok{"cost"}\NormalTok{) }\OperatorTok
\StringTok{      }\KeywordTok{add_min_set_objective}\NormalTok{() }\OperatorTok
\StringTok{      }\KeywordTok{add_relative_targets}\NormalTok{(}\FloatTok{0.1}\NormalTok{) }\OperatorTok
\StringTok{      }\KeywordTok{add_locked_in_constraints}\NormalTok{(}\StringTok{"locked_in"}\NormalTok{) }\OperatorTok
\StringTok{      }\KeywordTok{add_binary_decisions}\NormalTok{() }\OperatorTok
\StringTok{      }\KeywordTok{add_lpsymphony_solver}\NormalTok{()}

\CommentTok{# print problem}
\KeywordTok{print}\NormalTok{(p3)}
\end{Highlighting}
\end{Shaded}

\begin{verbatim}
## Conservation Problem
##   planning units: SpatialPolygonsDataFrame (557 units)
##   cost:           min: 2.55349, max: 47.23834
##   features:       vegetation.1, vegetation.2, vegetation.3, ... (33 features)
##   objective:      Minimum set objective 
##   targets:        Relative targets [targets (min: 0.1, max: 0.1)]
##   decisions:      Binary decision 
##   constraints:    <Locked in planning units [203 locked units]>
##   penalties:      <none>
##   portfolio:      default
##   solver:         Lpsymphony [first_feasible (0), gap (0.1), time_limit (-1), verbose (1)]
\end{verbatim}

\begin{Shaded}
\begin{Highlighting}[]
\CommentTok{# solve problem}
\NormalTok{s3 <-}\StringTok{ }\KeywordTok{solve}\NormalTok{(p3)}

\CommentTok{# plot solution}
\CommentTok{# selected = green, not selected = grey}
\KeywordTok{spplot}\NormalTok{(s3, }\StringTok{"solution_1"}\NormalTok{, }\DataTypeTok{col.regions =} \KeywordTok{c}\NormalTok{(}\StringTok{"grey80"}\NormalTok{, }\StringTok{"darkgreen"}\NormalTok{), }\DataTypeTok{main =} \StringTok{"s3"}\NormalTok{,}
       \DataTypeTok{colorkey =} \OtherTok{FALSE}\NormalTok{)}
\end{Highlighting}
\end{Shaded}

\begin{center}\includegraphics[width=0.65\linewidth]{prioritizr-workshop-manual_files/figure-latex/unnamed-chunk-36-1} \end{center}

Next, let's lock out highly degraded areas. Similar to before, this
information is present in our planning unit data so we can use the
\texttt{locked\_out} column name to achieve this.

\begin{Shaded}
\begin{Highlighting}[]
\CommentTok{# plot locked_out data}
\CommentTok{# TRUE = red, FALSE = grey}
\KeywordTok{spplot}\NormalTok{(pu_data, }\StringTok{"locked_out"}\NormalTok{, }\DataTypeTok{col.regions =} \KeywordTok{c}\NormalTok{(}\StringTok{"grey80"}\NormalTok{, }\StringTok{"darkred"}\NormalTok{),}
       \DataTypeTok{main =} \StringTok{"locked_out"}\NormalTok{, }\DataTypeTok{colorkey =} \OtherTok{FALSE}\NormalTok{)}
\end{Highlighting}
\end{Shaded}

\begin{center}\includegraphics[width=0.65\linewidth]{prioritizr-workshop-manual_files/figure-latex/unnamed-chunk-37-1} \end{center}

\begin{Shaded}
\begin{Highlighting}[]
\CommentTok{# make prioritization problem}
\NormalTok{p4 <-}\StringTok{ }\KeywordTok{problem}\NormalTok{(pu_data, veg_data, }\DataTypeTok{cost_column =} \StringTok{"cost"}\NormalTok{) }\OperatorTok
\StringTok{      }\KeywordTok{add_min_set_objective}\NormalTok{() }\OperatorTok
\StringTok{      }\KeywordTok{add_relative_targets}\NormalTok{(}\FloatTok{0.1}\NormalTok{) }\OperatorTok
\StringTok{      }\KeywordTok{add_locked_in_constraints}\NormalTok{(}\StringTok{"locked_in"}\NormalTok{) }\OperatorTok
\StringTok{      }\KeywordTok{add_locked_out_constraints}\NormalTok{(}\StringTok{"locked_out"}\NormalTok{) }\OperatorTok
\StringTok{      }\KeywordTok{add_binary_decisions}\NormalTok{() }\OperatorTok
\StringTok{      }\KeywordTok{add_lpsymphony_solver}\NormalTok{()}
\end{Highlighting}
\end{Shaded}

\begin{Shaded}
\begin{Highlighting}[]
\CommentTok{# print problem}
\KeywordTok{print}\NormalTok{(p4)}
\end{Highlighting}
\end{Shaded}

\begin{verbatim}
## Conservation Problem
##   planning units: SpatialPolygonsDataFrame (557 units)
##   cost:           min: 2.55349, max: 47.23834
##   features:       vegetation.1, vegetation.2, vegetation.3, ... (33 features)
##   objective:      Minimum set objective 
##   targets:        Relative targets [targets (min: 0.1, max: 0.1)]
##   decisions:      Binary decision 
##   constraints:    <Locked out planning units [6 locked units]
##                    Locked in planning units [203 locked units]>
##   penalties:      <none>
##   portfolio:      default
##   solver:         Lpsymphony [first_feasible (0), gap (0.1), time_limit (-1), verbose (1)]
\end{verbatim}

\begin{Shaded}
\begin{Highlighting}[]
\CommentTok{# solve problem}
\NormalTok{s4 <-}\StringTok{ }\KeywordTok{solve}\NormalTok{(p4)}

\CommentTok{# plot solution}
\CommentTok{# selected = green, not selected = grey}
\KeywordTok{spplot}\NormalTok{(s4, }\StringTok{"solution_1"}\NormalTok{, }\DataTypeTok{col.regions =} \KeywordTok{c}\NormalTok{(}\StringTok{"grey80"}\NormalTok{, }\StringTok{"darkgreen"}\NormalTok{), }\DataTypeTok{main =} \StringTok{"s4"}\NormalTok{,}
       \DataTypeTok{colorkey =} \OtherTok{FALSE}\NormalTok{)}
\end{Highlighting}
\end{Shaded}

\begin{center}\includegraphics[width=0.65\linewidth]{prioritizr-workshop-manual_files/figure-latex/unnamed-chunk-38-1} \end{center}

\clearpage

Now, let's compare the solutions.

\begin{rmdquestion} 1. What is the cost of the planning units
selected in \texttt{s2}, \texttt{s3}, and \texttt{s4}? 2. How many
planning units are in \texttt{s2}, \texttt{s3}, and \texttt{s4}? 3. Do
the solutions with more planning units have a greater cost? Why (or why
not)? 4. Why does the first solution (\texttt{s1}) cost less than the
second solution with protected areas locked into the solution
(\texttt{s2})? 5. Why does the third solution (\texttt{s3}) cost less
than the fourth solution solution with highly degraded areas locked out
(\texttt{s4})? \end{rmdquestion}

\section{Penalizing fragmentation}\label{penalizing-fragmentation}

Plans for protected area systems should promote connectivity. However,
the prioritizations we have made so far have been highly fragmented. To
address this issue, we can add penalties to our conservation planning
problem to penalize fragmentation. These penalties work by specifying a
trade-off between the primary objective (here, solution cost) and
fragmentation (i.e.~total exposed boundary length) using a penalty
value. If we set the penalty value too low, then we will end up with a
solution that is nearly identical to the previous solution. If we set
the penalty value too high, then prioritizr will (1) take a long time to
solve the problem and (2) we will end up with a solution that contains
lots of extra planning units that are not needed. This is because the
minimizing fragmentation is considered so much more important than
solution cost that the optimal solution is simply to select as many
planning units as possible.

As a rule of thumb, we generally want penalty values between 0.00001 and
0.01. However, finding a useful penalty value requires calibration. The
``correct'' penalty value depends on the size of the planning units, the
main objective values (e.g.~cost values), and the effect of
fragmentation on biodiversity persistence. Let's create a new problem
that is similar to our previous problem (\texttt{p4})---except that it
contains boundary length penalties and a slightly higher optimality gap
to reduce runtime (default is 0.1)---and solve it. Since our planning
unit data is in a spatial format (i.e.~vector or raster data),
prioritizr can automatically calculate the boundary data for us.

\clearpage

\begin{Shaded}
\begin{Highlighting}[]
\CommentTok{# make prioritization problem}
\NormalTok{p5 <-}\StringTok{ }\KeywordTok{problem}\NormalTok{(pu_data, veg_data, }\DataTypeTok{cost_column =} \StringTok{"cost"}\NormalTok{) }\OperatorTok
\StringTok{      }\KeywordTok{add_min_set_objective}\NormalTok{() }\OperatorTok
\StringTok{      }\KeywordTok{add_boundary_penalties}\NormalTok{(}\DataTypeTok{penalty =} \FloatTok{0.001}\NormalTok{) }\OperatorTok
\StringTok{      }\KeywordTok{add_relative_targets}\NormalTok{(}\FloatTok{0.1}\NormalTok{) }\OperatorTok
\StringTok{      }\KeywordTok{add_locked_in_constraints}\NormalTok{(}\StringTok{"locked_in"}\NormalTok{) }\OperatorTok
\StringTok{      }\KeywordTok{add_locked_out_constraints}\NormalTok{(}\StringTok{"locked_out"}\NormalTok{) }\OperatorTok
\StringTok{      }\KeywordTok{add_binary_decisions}\NormalTok{() }\OperatorTok
\StringTok{      }\KeywordTok{add_lpsymphony_solver}\NormalTok{()}

\CommentTok{# print problem}
\KeywordTok{print}\NormalTok{(p5)}
\end{Highlighting}
\end{Shaded}

\begin{verbatim}
## Conservation Problem
##   planning units: SpatialPolygonsDataFrame (557 units)
##   cost:           min: 2.55349, max: 47.23834
##   features:       vegetation.1, vegetation.2, vegetation.3, ... (33 features)
##   objective:      Minimum set objective 
##   targets:        Relative targets [targets (min: 0.1, max: 0.1)]
##   decisions:      Binary decision 
##   constraints:    <Locked in planning units [203 locked units]
##                    Locked out planning units [6 locked units]>
##   penalties:      <Boundary penalties [edge factor (min: 0.5, max: 0.5), penalty (0.001), zones]>
##   portfolio:      default
##   solver:         Lpsymphony [first_feasible (0), gap (0.1), time_limit (-1), verbose (1)]
\end{verbatim}

\begin{Shaded}
\begin{Highlighting}[]
\CommentTok{# solve problem,}
\CommentTok{# note this will take a bit longer than the previous runs}
\NormalTok{s5 <-}\StringTok{ }\KeywordTok{solve}\NormalTok{(p5)}

\CommentTok{# print solution}
\KeywordTok{print}\NormalTok{(s5)}
\end{Highlighting}
\end{Shaded}

\begin{verbatim}
## class       : SpatialPolygonsDataFrame 
## features    : 557 
## extent      : 1115191, 1385011, -4840595, -4662354  (xmin, xmax, ymin, ymax)
## crs         : +proj=aea +lat_1=-18 +lat_2=-36 +lat_0=0 +lon_0=132 +x_0=0 +y_0=0 +ellps=GRS80 +units=m +no_defs 
## variables   : 7
## names       :   id,            cost, status, locked_in, locked_out, pa_status, solution_1 
## min values  :  574, 2.5534941249227,      0,         0,          0,         0,          0 
## max values  : 1130, 47.238336402701,      2,         1,          1,         1,          1
\end{verbatim}

\begin{Shaded}
\begin{Highlighting}[]
\CommentTok{# plot solution}
\CommentTok{# selected = green, not selected = grey}
\KeywordTok{spplot}\NormalTok{(s5, }\StringTok{"solution_1"}\NormalTok{, }\DataTypeTok{col.regions =} \KeywordTok{c}\NormalTok{(}\StringTok{"grey80"}\NormalTok{, }\StringTok{"darkgreen"}\NormalTok{), }\DataTypeTok{main =} \StringTok{"s5"}\NormalTok{,}
       \DataTypeTok{colorkey =} \OtherTok{FALSE}\NormalTok{)}
\end{Highlighting}
\end{Shaded}

\begin{center}\includegraphics[width=0.65\linewidth]{prioritizr-workshop-manual_files/figure-latex/unnamed-chunk-41-1} \end{center}

Now let's compare the solutions to the problems with (\texttt{s5}) and
without (\texttt{s4}) the boundary length penalties.

\begin{rmdquestion} 1. What is the cost the fourth
(\texttt{s4}) and fifth (\texttt{s5}) solutions? Why does the fifth
solution (\texttt{s5}) cost more than the fourth (\texttt{s4}) solution?
2. Try setting the penalty value to 0.000000001 (i.e. \texttt{1e-9})
instead of 0.0005. What is the cost of the solution now? Is it different
from the fourth solution (\texttt{s4}) (hint: try plotting the solutions
to visualize them)? Is this is a useful penalty value? Why (or why not)?
3. Try setting the penalty value to 0.5. What is the cost of the
solution now? Is it different from the fourth solution (\texttt{s4})
(hint: try plotting the solutions to visualize them)? Is this a useful
penalty value? Why (or why not)?
\end{rmdquestion}

\chapter{Acknowledgements}\label{acknowledgements}

Many thanks to \href{https://icons8.com}{Icons8} for providing the icons
used in this manual and to Yihui Xie for developing the
\href{http://bookdown.org}{bookdown R package} that underpins this
manual. We also thank Garrett Grolemund and Hadley Wickham for creating
one of the Rstudio screenshots used in this manual that was originally a
part of their \emph{R for Data Science} book.

\chapter{Session information}\label{session-information}

\begin{Shaded}
\begin{Highlighting}[]
\CommentTok{# print session information}
\KeywordTok{sessionInfo}\NormalTok{()}
\end{Highlighting}
\end{Shaded}

\begin{verbatim}
## R version 4.0.0 (2020-04-24)
## Platform: x86_64-pc-linux-gnu (64-bit)
## Running under: Ubuntu 16.04.6 LTS
## 
## Matrix products: default
## BLAS:   /home/travis/R-bin/lib/R/lib/libRblas.so
## LAPACK: /home/travis/R-bin/lib/R/lib/libRlapack.so
## 
## locale:
##  [1] LC_CTYPE=en_US.UTF-8       LC_NUMERIC=C              
##  [3] LC_TIME=en_US.UTF-8        LC_COLLATE=en_US.UTF-8    
##  [5] LC_MONETARY=en_US.UTF-8    LC_MESSAGES=en_US.UTF-8   
##  [7] LC_PAPER=en_US.UTF-8       LC_NAME=C                 
##  [9] LC_ADDRESS=C               LC_TELEPHONE=C            
## [11] LC_MEASUREMENT=en_US.UTF-8 LC_IDENTIFICATION=C       
## 
## attached base packages:
## [1] stats     graphics  grDevices utils     datasets  methods   base     
## 
## other attached packages:
##  [1] dplyr_1.0.0      gridExtra_2.3    assertthat_0.2.1 scales_1.1.1    
##  [5] units_0.6-7      rgeos_0.5-3      rgdal_1.5-12     prioritizr_5.0.1
##  [9] proto_1.0.0      sf_0.9-5         raster_3.3-13    sp_1.4-2        
## 
## loaded via a namespace (and not attached):
##  [1] Rcpp_1.0.5          compiler_4.0.0      pillar_1.4.6       
##  [4] class_7.3-16        tools_4.0.0         uuid_0.1-4         
##  [7] digest_0.6.25       gtable_0.3.0        evaluate_0.14      
## [10] lifecycle_0.2.0     tibble_3.0.3        lattice_0.20-41    
## [13] pkgconfig_2.0.3     rlang_0.4.7         Matrix_1.2-18      
## [16] lpsymphony_1.16.0   cli_2.0.2           DBI_1.1.0          
## [19] parallel_4.0.0      yaml_2.2.1          xfun_0.15          
## [22] e1071_1.7-3         exactextractr_0.4.0 stringr_1.4.0      
## [25] knitr_1.29          generics_0.0.2      vctrs_0.3.2        
## [28] classInt_0.4-3      grid_4.0.0          tidyselect_1.1.0   
## [31] glue_1.4.1          R6_2.4.1            fansi_0.4.1        
## [34] rmarkdown_2.3       bookdown_0.20.1     purrr_0.3.4        
## [37] magrittr_1.5        codetools_0.2-16    htmltools_0.5.0    
## [40] ellipsis_0.3.1      colorspace_1.4-1    utf8_1.1.4         
## [43] KernSmooth_2.23-16  stringi_1.4.6       munsell_0.5.0      
## [46] crayon_1.3.4
\end{verbatim}

\chapter{References}\label{references}

\bibliography{references.bib}

\end{document}
